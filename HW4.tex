\documentclass[11pt]{article}
\usepackage{writeup/common}
\usepackage{url}
\usepackage{listings}

\title{CS 287 \\ Assignment 4: Word Segmentation}
\date{}
\begin{document}

\maketitle{}

\begin{center}
  \textbf{Due:} Friday, April 1st, 11:59 pm 
\end{center}


Over the last couple years there has been tremendous interest in
recurrent neural networks for natural language tasks, and in
particular the long short-term memory (LSTM) network. The last 
several classes have focused on the theory and structure of 
recurrent networks. For this assignment you will actually 
construct utilize an LSTM for a real-world problem. 

The problem we will focus on is word segmentation, that is identifying
the spaces in sentence-based on the previous characters only.  This is
a simplified version of word segmentation processes necessary for
processing languages written without spaces, such as Korean or Chinese.
Unlike past assignments, our work here will not be based on any
existing paper. However this will give you a chance 
to experiment with the LSTM on this task. 

Before you start we advise that you get very comfortable with the
notes from class, the papers on LSTMs, and especially the torch
\texttt{rnn} library.

As you complete this assignment, we ask that you submit your results
on the test data to the Kaggle competition website at
\url{https://inclass.kaggle.com/c/cs287-hw4} and that you compile your
experiences in a write-up based on the template at
\url{https://github.com/cs287/hw_template}.

\section{Data and Preprocessing}

\subsection{Data}

The data for this task is under the \texttt{data/} directory. 
Our data is the same of the previous assignment, except that 
now we have given you only the characters of the data set 
separated with explicit space tokens. 


\lstset{ basicstyle=\ttfamily, breaklines=true}

\begin{lstlisting}
> head data/valid_chars.txt 
c o n s u m e r s <space> m a y <space> w a n t <space> t o <space> m o v e <space> t h e i r
<space> t e l e p h o n e s <space> a <space> l i t t l e <space> c l o s e r <space> t o <space> 
t h e <space> t v <space> s e t </s> < u n k > <space> < u n k > <space> w a t c h i n g <space> 
a b c <space> ' s <space> m o n d a y <space> n i g h t <space> f o o t b a l l <space> c a n 
<space> n o w <space> v o t e <space> d u r i n g <space> < u n k > <space> f o r <space> t h e 
<space> g r e a t e s t <space> p l a y <space> i n <space> N <space> y e a r s <space> f r o m 
<space> a m o n g <space> f o u r <space> o r <space> f i v e <space> < u n k > <sp
\end{lstlisting}

Note that for this problem set we no longer print the data file in lines, we 
simply concatenate together each one of the sentence separated with a $</s>$
symbol. 


The test data is in file \texttt{test\_chars.txt}. This file consists 
of one sentence per line and includes no explicit spaces. The aim of 
the project is to re-insert spaces back into the sentences. We have 
also provided this file for the validation data.

\begin{lstlisting}
> head data/valid_chars_kaggle.txt 
c o n s u m e r s  m a y  w a n t  t o  m o v e  t h e i r  t e l e p h o n e s  a  l i t t l 
e  c l o s e r  t o  t h e  t v  s e t </s>                                                  
< u n k >  < u n k >  w a t c h i n g  a b c  ' s  m o n d a y  n i g h t  f o o t b a l l  c 
a n  n o w  v o t e  d u r i n g  < u n k >  f o r  t h e  g r e a t e s t  p l a y  i n  N  y
 e a r s  f r o m  a m o n g  f o u r  o r  f i v e  < u n k >  < u n k > </s>               
t w o  w e e k s  a g o  v i e w e r s  o f  s e v e r a l  n b c  < u n k >  c o n s u m e r 
 s e g m e n t s  s t a r t e d  c a l l i n g  a  N  n u m b e r  f o r  a d v i c e  o n  v 
a r i o u s  < u n k >  i s s u e s </s>                                                     
a n d  t h e  n e w  s y n d i c a t e d  r e a l i t y  s h o w  h a r d  c o p y  r e c o r 
d s  v i e w e r s  '  o p i n i o n s  f o r  p o s s i b l e  a i r i n g  o n  t h e  n e x
 t  d a y  ' s  s h o w </s>                                                                 
i n t e r a c t i v e  t e l e p h o n e  t e c h n o l o g y  h a s  t a k e n  a  n e w  l e
\end{lstlisting}

The Kaggle answer file should simply provide the number of spaces for each sentence. 
We have provided you with a sample answer file for reference.     

\begin{lstlisting}
> head data/valid_chars_kaggle_answer.txt 
ID,Count
1,13
2,26
3,20
4,20
5,18
6,11
7,26
\end{lstlisting}

\subsection{Preprocessing}

For this assignment you will write your own preprocessing code
in \texttt{preprocessing.py}. Preprocessing for this assignment
means transforming the training corpus into a representation 
usable for RNN training. As in previous assignments you 
will need to create a map from characters 
to indices. However, since there is no longer a fixed window size the layout is 
a bit different. The following is a description of how to structure the 
input data for a transducer RNN.

Say our training corpus consists of character $w_1, \ldots, w_n$ 
In a perfect world, we could have the RNN as one long matrix. 

\[
  \begin{bmatrix}
    w_1 & w_2 & \ldots & w_n
  \end{bmatrix}
\] 

However, this would mean we would need to wait $n$ steps to backprop. Instead 
it is more common to select a fixed \textit{sequence length} $l$ for doing 
backpropagation. We pass the hidden state  between sequences ($\bolds_l$ combines with $w_{l+1}$), but only 
backprop within a sequence length window. 

\[
  \begin{bmatrix}
    w_1 & w_2 & \ldots & w_{l}
  \end{bmatrix}   \begin{bmatrix}
    w_{l+1} & w_{l+2} & \ldots & w_{2l}
  \end{bmatrix}
  \ldots
  \begin{bmatrix}
    w_{n-l+1} & w_{n-l+2} & \ldots & w_{n}
  \end{bmatrix}
\] 

When you use this strategy, you much call the following rnn function to save states between 
matrices 

\begin{lstlisting}
model:remember('both')
\end{lstlisting}

While this method has a fixed backprop length of $s$, with a reasonable 
size $s$ it works well in practice. Unfortunately it is quite slow since 
we cannot use a batch (sequential processing). To get around this issue 
we split the sequence into $b$ parts and fold them over each other. 



\[
  \begin{bmatrix}
    w_1 & w_2 & \ldots & w_{l}\\
    w_{n/b+1} & w_{n/b+1} & \ldots & w_{n/b+l}\\
    \vdots \\
    w_{(b-1)n/b + 1} & w_{(b-1)n/b+2} & \ldots & w_{(b-1)n/b+l}
  \end{bmatrix}   \begin{bmatrix}
    w_{l+1} &  \ldots & w_{2l} \\
    w_{n/b+l+1} &  \ldots & w_{n/b+2l}\\
    & \vdots & \\
    & \ldots &  
  \end{bmatrix}
  \ldots
  \begin{bmatrix}
    w_{n/b-l+1} & w_{n/b-l+2} & \ldots & w_{n/b}\\
    & \ldots & \\
    & \vdots & \\
    & \ldots & 
  \end{bmatrix}
\] 

This allows both batching and backprop with a fixed length. Note
though that the hidden states at first timestep ($\bolds_{n/b+1}$)
will be incorrect since they get computed before their correct
previous hidden ($\bolds_{n/b}$). However this is tolerable since it
only leads to $b$ incorrect hidden states during training.




Since we will be building a transducer, we also need the training 
data to be in the same format. Say we are predicting $\hat{y}$ at 
each step. Then our target output will be of the form, 

\[
  \begin{bmatrix}
    y_1 & y_2 & \ldots & y_{l}\\
    y_{n/b+1} & y_{n/b+1} & \ldots & y_{n/b+l}\\
    \vdots \\
    y_{(b-1)n/b + 1} & y_{(b-1)n/b+2} & \ldots & y_{(b-1)n/b+l}
  \end{bmatrix}   \begin{bmatrix}
    y_{l+1} &  \ldots & y_{2l} \\
    y_{n/b+l+1} &  \ldots & y_{n/b+2l}\\
    & \vdots & \\
    & \ldots &  
  \end{bmatrix}
  \ldots
  \begin{bmatrix}
    y_{n/b-l+1} & y_{n/b-l+2} & \ldots & y_{n/b}\\
    & \ldots & \\
    & \vdots & \\
    & \ldots & 
  \end{bmatrix}
\] 

For language modeling the target output $y_i$ would be the next word
$w_{i+1}$.  For this assignment, our goal is simply space
prediction. Therefore the target output $y_i$ will be $2$ if the next
word $w_{i+1}$ is a space and $1$ otherwise.

\section{Code Setup}

Write your main code in \texttt{HW4.lua}. For this assignment part,
you can (and should) use the \texttt{nn} and \texttt{rnn} library in addition to the
standard Torch library. You should not need to use any extra libraries.

\subsection{Hyperparameters}

Several of the models described have explicit hyperparameters that you will 
need to tune. It is your responsibility to cleanly separate these out from 
the models themselves and expose as command-line options. This makes it much 
easier to run experiments and to utilize experimental scripts. 

\subsection{Logging and Reporting}

As part of the write-up, you will need to report on the training and
predictive accuracy of your models. To make this possible, your code
should report on various metrics of the model both at training and
test time. We will leave it up to you on which metrics to log, but we
recommend reporting training speed, training set NLL, training set
predictive accuracy, and validation predictive accuracy. It is your
responsibility to convince us that the model is correctly training.

\section{Models}

For this assignment you will implement the three models. We will warm
up with a count-based model. Then we will
try our Bengio NNLM and an LSTM model. 


\subsection{Count-Based Model}

To start, modify your code from HW3
to implement a count-based character n-gram model. 
The model should only give the probability of the next 
word being a space  

\[ P(w_i =<\mathrm{space}> | w_{i-n+1}, \ldots w_{i-1}) \]

Test the model out by determining its perplexity on 
the validation set and by running on the segmentation task. 

For segmentation you should run two inference algorithms. 

\begin{enumerate}
\item Greedily predict whether the next word is a space, and feedback 
  into the model. (i.e. if it is a space, add the $<$space$>$ to the context. Otherwise continue normally).

\item Dynamic programming over n-grams as described in class. 
\end{enumerate}

\subsection{Neural Language Model}

As a second baseline, you will modify your neural language 
model from the HW3 to predict whether the next character is a 
space or not. This should be relatively similar to your previous 
code, except that you will be prediction $1$ or $2$ instead 
of a softmax over words. For this problem you can have a much smaller embedding 
size of around $\approx 15$ 

As above you should test your model 
by running greedy search and dynamic programming (for a small size). 

\subsection{RNN Model}

The main portion of the assignment will be implementing the 
RNN model using the Elements \texttt{rnn} library. Much of the 
main portion of the work is done for you by the library. However 
getting the details right is crucially important. 

For this model you should implement a generic RNN transducer. 
Given the current state of the RNN you are simply trying to predict 
whether this is a space in the next word. 

\paragraph{Necessary Hacks:}
In any torch model you can call the following function one time to get 
a vector of all the params and gradients of the params. 

\begin{lstlisting}
   local params, grad_params = model:getParameters()
\end{lstlisting}

Once you have these vectors you should do the following:
\begin{itemize}
\item To begin you should initialize all the params to be uniform between $-0.05$ and $0.05$.
\item Before updating your parameters, you should perform gradient normalization with max norm $5$.
\end{itemize}

Once you have the model, use it in the greedy fashion described above. 
At each time step predict whether the next word is a space. If it is, feed 
in a $<$space$>$ to the model before continuing, otherwise continue onward. 
Keep track of how many spaces you have seen so far. 

\subsection{Additional Experiments}

Once these models are constructed, you should also report on
additional experiments on these data sets. We will leave this aspect
open-ended, but suggestion include:

\begin{itemize}
\item Compare performance between the LSTM, RNN and GRU on this task

\item Experiment with stack LSTM and dropout. The rnn library provides details for how to implement this. Also see \cite{}. 
\item Try implementing a bidirectional LSTM for prediction. This model is provided by the rnn library. In this model you do not need to do greedy search, however you will 
  have to change your training such that you do not see the spaces in the data.
\item Experiment with different optimization techniques. For instance see the \texttt{optim} package. 

\end{itemize}

\section{Report and Submission}

For your write-up, follow the report template at
\url{https://github.com/cs287/hw_template}. Be sure to include a link
to your code, Kaggle ID, and reports on your results.

In addition to submitting your Kaggle results, we also expect you to report on your 
experimental process. This should include data tables, graphs and discussion of any 
issues that you may run into. 

\bibliographystyle{apalike} 
\bibliography{HW4}

\end{document}
